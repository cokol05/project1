%% Generated by Sphinx.
\def\sphinxdocclass{report}
\documentclass[letterpaper,10pt,russian]{sphinxmanual}
\ifdefined\pdfpxdimen
   \let\sphinxpxdimen\pdfpxdimen\else\newdimen\sphinxpxdimen
\fi \sphinxpxdimen=.75bp\relax
\ifdefined\pdfimageresolution
    \pdfimageresolution= \numexpr \dimexpr1in\relax/\sphinxpxdimen\relax
\fi
%% let collapsible pdf bookmarks panel have high depth per default
\PassOptionsToPackage{bookmarksdepth=5}{hyperref}

\PassOptionsToPackage{booktabs}{sphinx}
\PassOptionsToPackage{colorrows}{sphinx}

\PassOptionsToPackage{warn}{textcomp}
\usepackage[utf8]{inputenc}
\ifdefined\DeclareUnicodeCharacter
% support both utf8 and utf8x syntaxes
  \ifdefined\DeclareUnicodeCharacterAsOptional
    \def\sphinxDUC#1{\DeclareUnicodeCharacter{"#1}}
  \else
    \let\sphinxDUC\DeclareUnicodeCharacter
  \fi
  \sphinxDUC{00A0}{\nobreakspace}
  \sphinxDUC{2500}{\sphinxunichar{2500}}
  \sphinxDUC{2502}{\sphinxunichar{2502}}
  \sphinxDUC{2514}{\sphinxunichar{2514}}
  \sphinxDUC{251C}{\sphinxunichar{251C}}
  \sphinxDUC{2572}{\textbackslash}
\fi
\usepackage{cmap}
\usepackage[T1]{fontenc}
\usepackage{amsmath,amssymb,amstext}
\usepackage{babel}





\usepackage[Sonny]{fncychap}
\ChNameVar{\Large\normalfont\sffamily}
\ChTitleVar{\Large\normalfont\sffamily}
\usepackage{sphinx}

\fvset{fontsize=auto}
\usepackage{geometry}


% Include hyperref last.
\usepackage{hyperref}
% Fix anchor placement for figures with captions.
\usepackage{hypcap}% it must be loaded after hyperref.
% Set up styles of URL: it should be placed after hyperref.
\urlstyle{same}

\addto\captionsrussian{\renewcommand{\contentsname}{Содержание:}}

\usepackage{sphinxmessages}
\setcounter{tocdepth}{1}



\title{Console task manager}
\date{нояб. 15, 2025}
\release{1.0.0}
\author{Anton}
\newcommand{\sphinxlogo}{\vbox{}}
\renewcommand{\releasename}{Выпуск}
\makeindex
\begin{document}

\ifdefined\shorthandoff
  \ifnum\catcode`\=\string=\active\shorthandoff{=}\fi
  \ifnum\catcode`\"=\active\shorthandoff{"}\fi
\fi

\pagestyle{empty}
\sphinxmaketitle
\pagestyle{plain}
\sphinxtableofcontents
\pagestyle{normal}
\phantomsection\label{\detokenize{index::doc}}


\sphinxAtStartPar
Добро пожаловать в документацию по \sphinxstylestrong{Консольному менеджеру задач}!

\sphinxstepscope


\chapter{Модуль taskmanager}
\label{\detokenize{taskmanager:module-taskmanager}}\label{\detokenize{taskmanager:taskmanager}}\label{\detokenize{taskmanager::doc}}\index{module@\spxentry{module}!taskmanager@\spxentry{taskmanager}}\index{taskmanager@\spxentry{taskmanager}!module@\spxentry{module}}
\sphinxAtStartPar
Существующие команды для main.py:
usage: main.py {[}\sphinxhyphen{}h{]} \{add,list,done,delete,view\} …
\begin{description}
\sphinxlineitem{positional arguments:}\begin{description}
\sphinxlineitem{\{add,list,done,delete,view\}}\begin{quote}

\sphinxAtStartPar
Команды
\end{quote}

\sphinxAtStartPar
add                 Добавить новую задачу
list                Показать список задач
done                Отметить задачу как выполненную
delete              Удалить задачу
view                Просмотреть детали задачи

\end{description}

\sphinxlineitem{options:}\begin{optionlist}{3cm}
\item [\sphinxhyphen{}h, \sphinxhyphen{}\sphinxhyphen{}help]  
\sphinxAtStartPar
show this help message and exit
\end{optionlist}

\end{description}

\sphinxAtStartPar
Подробное описание команды add:
usage: main.py add {[}\sphinxhyphen{}h{]} {[}\textendash{}description DESCRIPTION{]} {[}\textendash{}priority \{low,medium,high\}{]} {[}\textendash{}due\sphinxhyphen{}date DUE\_DATE{]} title
\begin{description}
\sphinxlineitem{positional arguments:}
\sphinxAtStartPar
title                 Название задачи

\sphinxlineitem{options:}\begin{optionlist}{3cm}
\item [\sphinxhyphen{}h, \sphinxhyphen{}\sphinxhyphen{}help]  
\sphinxAtStartPar
show this help message and exit
\item [\sphinxhyphen{}\sphinxhyphen{}description, \sphinxhyphen{}d DESCRIPTION]  
\sphinxAtStartPar
Описание задачи
\end{optionlist}
\begin{description}
\sphinxlineitem{\textendash{}priority, \sphinxhyphen{}p \{low,medium,high\}}
\sphinxAtStartPar
Приоритет задачи

\end{description}
\begin{optionlist}{3cm}
\item [\sphinxhyphen{}\sphinxhyphen{}due\sphinxhyphen{}date DUE\_DATE]  
\sphinxAtStartPar
Дата выполнения (YYYY\sphinxhyphen{}MM\sphinxhyphen{}DD)
\end{optionlist}

\end{description}

\sphinxAtStartPar
Подробное описание команды list:
usage: main.py list {[}\sphinxhyphen{}h{]} {[}\textendash{}status \{Ожидание,Выполнено\}{]} {[}\textendash{}priority \{low,medium,high\}{]} {[}\textendash{}hide\sphinxhyphen{}completed{]}
\begin{description}
\sphinxlineitem{options:}\begin{optionlist}{3cm}
\item [\sphinxhyphen{}h, \sphinxhyphen{}\sphinxhyphen{}help]  
\sphinxAtStartPar
show this help message and exit
\end{optionlist}
\begin{description}
\sphinxlineitem{\textendash{}status \{Ожидание,Выполнено\}}
\sphinxAtStartPar
Фильтр по статусу

\sphinxlineitem{\textendash{}priority \{low,medium,high\}}
\sphinxAtStartPar
Фильтр по приоритету

\end{description}
\begin{optionlist}{3cm}
\item [\sphinxhyphen{}\sphinxhyphen{}hide\sphinxhyphen{}completed]  
\sphinxAtStartPar
Скрыть выполненные задачи
\end{optionlist}

\end{description}

\sphinxAtStartPar
Подробное описание команды done:
usage: main.py done {[}\sphinxhyphen{}h{]} task\_id
\begin{description}
\sphinxlineitem{positional arguments:}
\sphinxAtStartPar
task\_id     ID задачи

\sphinxlineitem{options:}\begin{optionlist}{3cm}
\item [\sphinxhyphen{}h, \sphinxhyphen{}\sphinxhyphen{}help]  
\sphinxAtStartPar
show this help message and exit
\end{optionlist}

\end{description}

\sphinxAtStartPar
Подробное описание команды delete:
usage: main.py delete {[}\sphinxhyphen{}h{]} task\_id
\begin{description}
\sphinxlineitem{positional arguments:}
\sphinxAtStartPar
task\_id     ID задачи

\sphinxlineitem{options:}\begin{optionlist}{3cm}
\item [\sphinxhyphen{}h, \sphinxhyphen{}\sphinxhyphen{}help]  
\sphinxAtStartPar
show this help message and exit
\end{optionlist}

\end{description}

\sphinxAtStartPar
Подробное описание команды view:
usage: main.py view {[}\sphinxhyphen{}h{]} task\_id
\begin{description}
\sphinxlineitem{positional arguments:}
\sphinxAtStartPar
task\_id     ID задачи

\sphinxlineitem{options:}\begin{optionlist}{3cm}
\item [\sphinxhyphen{}h, \sphinxhyphen{}\sphinxhyphen{}help]  
\sphinxAtStartPar
show this help message and exit
\end{optionlist}

\end{description}


\section{Submodules}
\label{\detokenize{taskmanager:submodules}}
\sphinxstepscope


\subsection{taskmanager.models module}
\label{\detokenize{taskmanager.models:module-taskmanager.models}}\label{\detokenize{taskmanager.models:taskmanager-models-module}}\label{\detokenize{taskmanager.models::doc}}\index{module@\spxentry{module}!taskmanager.models@\spxentry{taskmanager.models}}\index{taskmanager.models@\spxentry{taskmanager.models}!module@\spxentry{module}}
\sphinxAtStartPar
Этот файл содержит класс Task, который описывает структуру одной задачи.
Здесь определяются, какие данные хранятся в каждой задаче.
\index{Task (класс в taskmanager.models)@\spxentry{Task}\spxextra{класс в taskmanager.models}}

\begin{fulllineitems}
\phantomsection\label{\detokenize{taskmanager.models:taskmanager.models.Task}}
\pysigstartsignatures
\pysiglinewithargsret
{\sphinxbfcode{\sphinxupquote{\DUrole{k}{class}\DUrole{w}{ }}}\sphinxcode{\sphinxupquote{taskmanager.models.}}\sphinxbfcode{\sphinxupquote{Task}}}
{\sphinxparam{\DUrole{n}{title}}\sphinxparamcomma \sphinxparam{\DUrole{n}{description}}\sphinxparamcomma \sphinxparam{\DUrole{n}{priority}}\sphinxparamcomma \sphinxparam{\DUrole{n}{due\_date}\DUrole{o}{=}\DUrole{default_value}{None}}\sphinxparamcomma \sphinxparam{\DUrole{n}{id}\DUrole{o}{=}\DUrole{default_value}{None}}\sphinxparamcomma \sphinxparam{\DUrole{n}{status}\DUrole{o}{=}\DUrole{default_value}{None}}\sphinxparamcomma \sphinxparam{\DUrole{n}{created\_date}\DUrole{o}{=}\DUrole{default_value}{None}}\sphinxparamcomma \sphinxparam{\DUrole{n}{completed\_date}\DUrole{o}{=}\DUrole{default_value}{None}}}
{}
\pysigstopsignatures
\sphinxAtStartPar
Базовые классы: \sphinxcode{\sphinxupquote{object}}

\sphinxAtStartPar
Класс служит для описания структуры одной задачи.
\index{id (атрибут taskmanager.models.Task)@\spxentry{id}\spxextra{атрибут taskmanager.models.Task}}

\begin{fulllineitems}
\phantomsection\label{\detokenize{taskmanager.models:taskmanager.models.Task.id}}
\pysigstartsignatures
\pysigline
{\sphinxbfcode{\sphinxupquote{id}}}
\pysigstopsignatures
\sphinxAtStartPar
Уникальный идентификатор задачи
\begin{quote}\begin{description}
\sphinxlineitem{Type}
\sphinxAtStartPar
int, optional

\end{description}\end{quote}

\end{fulllineitems}

\index{title (атрибут taskmanager.models.Task)@\spxentry{title}\spxextra{атрибут taskmanager.models.Task}}

\begin{fulllineitems}
\phantomsection\label{\detokenize{taskmanager.models:taskmanager.models.Task.title}}
\pysigstartsignatures
\pysigline
{\sphinxbfcode{\sphinxupquote{title}}}
\pysigstopsignatures
\sphinxAtStartPar
Название задачи
\begin{quote}\begin{description}
\sphinxlineitem{Type}
\sphinxAtStartPar
str

\end{description}\end{quote}

\end{fulllineitems}

\index{description (атрибут taskmanager.models.Task)@\spxentry{description}\spxextra{атрибут taskmanager.models.Task}}

\begin{fulllineitems}
\phantomsection\label{\detokenize{taskmanager.models:taskmanager.models.Task.description}}
\pysigstartsignatures
\pysigline
{\sphinxbfcode{\sphinxupquote{description}}}
\pysigstopsignatures
\sphinxAtStartPar
Подробное описание задачи
\begin{quote}\begin{description}
\sphinxlineitem{Type}
\sphinxAtStartPar
str

\end{description}\end{quote}

\end{fulllineitems}

\index{status (атрибут taskmanager.models.Task)@\spxentry{status}\spxextra{атрибут taskmanager.models.Task}}

\begin{fulllineitems}
\phantomsection\label{\detokenize{taskmanager.models:taskmanager.models.Task.status}}
\pysigstartsignatures
\pysigline
{\sphinxbfcode{\sphinxupquote{status}}}
\pysigstopsignatures
\sphinxAtStartPar
Статус выполнения задачи
\begin{quote}\begin{description}
\sphinxlineitem{Type}
\sphinxAtStartPar
str

\end{description}\end{quote}

\end{fulllineitems}

\index{priority (атрибут taskmanager.models.Task)@\spxentry{priority}\spxextra{атрибут taskmanager.models.Task}}

\begin{fulllineitems}
\phantomsection\label{\detokenize{taskmanager.models:taskmanager.models.Task.priority}}
\pysigstartsignatures
\pysigline
{\sphinxbfcode{\sphinxupquote{priority}}}
\pysigstopsignatures
\sphinxAtStartPar
Уровень приоритета задачи
\begin{quote}\begin{description}
\sphinxlineitem{Type}
\sphinxAtStartPar
str

\end{description}\end{quote}

\end{fulllineitems}

\index{created\_date (атрибут taskmanager.models.Task)@\spxentry{created\_date}\spxextra{атрибут taskmanager.models.Task}}

\begin{fulllineitems}
\phantomsection\label{\detokenize{taskmanager.models:taskmanager.models.Task.created_date}}
\pysigstartsignatures
\pysigline
{\sphinxbfcode{\sphinxupquote{created\_date}}}
\pysigstopsignatures
\sphinxAtStartPar
Дата создания в формате ISO
\begin{quote}\begin{description}
\sphinxlineitem{Type}
\sphinxAtStartPar
str

\end{description}\end{quote}

\end{fulllineitems}

\index{due\_date (атрибут taskmanager.models.Task)@\spxentry{due\_date}\spxextra{атрибут taskmanager.models.Task}}

\begin{fulllineitems}
\phantomsection\label{\detokenize{taskmanager.models:taskmanager.models.Task.due_date}}
\pysigstartsignatures
\pysigline
{\sphinxbfcode{\sphinxupquote{due\_date}}}
\pysigstopsignatures
\sphinxAtStartPar
Дата выполнения в формате ISO
\begin{quote}\begin{description}
\sphinxlineitem{Type}
\sphinxAtStartPar
str, optional

\end{description}\end{quote}

\end{fulllineitems}

\index{completed\_date (атрибут taskmanager.models.Task)@\spxentry{completed\_date}\spxextra{атрибут taskmanager.models.Task}}

\begin{fulllineitems}
\phantomsection\label{\detokenize{taskmanager.models:taskmanager.models.Task.completed_date}}
\pysigstartsignatures
\pysigline
{\sphinxbfcode{\sphinxupquote{completed\_date}}}
\pysigstopsignatures
\sphinxAtStartPar
Дата завершения в формате ISO
\begin{quote}\begin{description}
\sphinxlineitem{Type}
\sphinxAtStartPar
str, optional

\end{description}\end{quote}

\end{fulllineitems}

\index{\_\_init\_\_() (метод taskmanager.models.Task)@\spxentry{\_\_init\_\_()}\spxextra{метод taskmanager.models.Task}}

\begin{fulllineitems}
\phantomsection\label{\detokenize{taskmanager.models:taskmanager.models.Task.__init__}}
\pysigstartsignatures
\pysiglinewithargsret
{\sphinxbfcode{\sphinxupquote{\_\_init\_\_}}}
{\sphinxparam{\DUrole{n}{title}}\sphinxparamcomma \sphinxparam{\DUrole{n}{description}}\sphinxparamcomma \sphinxparam{\DUrole{n}{priority}}\sphinxparamcomma \sphinxparam{\DUrole{n}{due\_date}\DUrole{o}{=}\DUrole{default_value}{None}}\sphinxparamcomma \sphinxparam{\DUrole{n}{id}\DUrole{o}{=}\DUrole{default_value}{None}}\sphinxparamcomma \sphinxparam{\DUrole{n}{status}\DUrole{o}{=}\DUrole{default_value}{None}}\sphinxparamcomma \sphinxparam{\DUrole{n}{created\_date}\DUrole{o}{=}\DUrole{default_value}{None}}\sphinxparamcomma \sphinxparam{\DUrole{n}{completed\_date}\DUrole{o}{=}\DUrole{default_value}{None}}}
{}
\pysigstopsignatures
\sphinxAtStartPar
Метод инициализирует необходимые атрибуты.
\begin{quote}\begin{description}
\sphinxlineitem{Параметры}\begin{itemize}
\item {} 
\sphinxAtStartPar
\sphinxstyleliteralstrong{\sphinxupquote{title}} (\sphinxstyleliteralemphasis{\sphinxupquote{str}}) \textendash{} Название задачи;

\item {} 
\sphinxAtStartPar
\sphinxstyleliteralstrong{\sphinxupquote{description}} (\sphinxstyleliteralemphasis{\sphinxupquote{str}}) \textendash{} Подробное описание задачи;

\item {} 
\sphinxAtStartPar
\sphinxstyleliteralstrong{\sphinxupquote{priority}} (\sphinxstyleliteralemphasis{\sphinxupquote{str}}) \textendash{} Уровень приоритета („low“, „medium“, „high“);

\item {} 
\sphinxAtStartPar
\sphinxstyleliteralstrong{\sphinxupquote{due\_date}} (\sphinxstyleliteralemphasis{\sphinxupquote{str}}\sphinxstyleliteralemphasis{\sphinxupquote{, }}\sphinxstyleliteralemphasis{\sphinxupquote{optional}}) \textendash{} Дата выполнения в формате ISO. По умолчанию None;

\item {} 
\sphinxAtStartPar
\sphinxstyleliteralstrong{\sphinxupquote{id}} (\sphinxstyleliteralemphasis{\sphinxupquote{int}}\sphinxstyleliteralemphasis{\sphinxupquote{, }}\sphinxstyleliteralemphasis{\sphinxupquote{optional}}) \textendash{} Уникальный идентификатор. По умолчанию None;

\item {} 
\sphinxAtStartPar
\sphinxstyleliteralstrong{\sphinxupquote{status}} (\sphinxstyleliteralemphasis{\sphinxupquote{str}}\sphinxstyleliteralemphasis{\sphinxupquote{, }}\sphinxstyleliteralemphasis{\sphinxupquote{optional}}) \textendash{} Статус выполнения. По умолчанию „Ожидание“;

\item {} 
\sphinxAtStartPar
\sphinxstyleliteralstrong{\sphinxupquote{created\_date}} (\sphinxstyleliteralemphasis{\sphinxupquote{str}}\sphinxstyleliteralemphasis{\sphinxupquote{, }}\sphinxstyleliteralemphasis{\sphinxupquote{optional}}) \textendash{} Дата создания. По умолчанию текущая дата;

\item {} 
\sphinxAtStartPar
\sphinxstyleliteralstrong{\sphinxupquote{completed\_date}} (\sphinxstyleliteralemphasis{\sphinxupquote{str}}\sphinxstyleliteralemphasis{\sphinxupquote{, }}\sphinxstyleliteralemphasis{\sphinxupquote{optional}}) \textendash{} Дата завершения. По умолчанию None.

\end{itemize}

\end{description}\end{quote}

\end{fulllineitems}

\index{\_\_str\_\_() (метод taskmanager.models.Task)@\spxentry{\_\_str\_\_()}\spxextra{метод taskmanager.models.Task}}

\begin{fulllineitems}
\phantomsection\label{\detokenize{taskmanager.models:taskmanager.models.Task.__str__}}
\pysigstartsignatures
\pysiglinewithargsret
{\sphinxbfcode{\sphinxupquote{\_\_str\_\_}}}
{}
{}
\pysigstopsignatures
\sphinxAtStartPar
Метод отображает информацию о задачах.
\begin{quote}\begin{description}
\sphinxlineitem{Результат}
\sphinxAtStartPar
Строка с форматированием, показывающая основную информацию о задаче.

\sphinxlineitem{Тип результата}
\sphinxAtStartPar
str

\end{description}\end{quote}

\end{fulllineitems}

\index{change\_task\_execution\_status() (метод taskmanager.models.Task)@\spxentry{change\_task\_execution\_status()}\spxextra{метод taskmanager.models.Task}}

\begin{fulllineitems}
\phantomsection\label{\detokenize{taskmanager.models:taskmanager.models.Task.change_task_execution_status}}
\pysigstartsignatures
\pysiglinewithargsret
{\sphinxbfcode{\sphinxupquote{change\_task\_execution\_status}}}
{}
{}
\pysigstopsignatures
\sphinxAtStartPar
Метод изменяет статус задачи на „Выполнено“ и устанавливает текущую дату и время, записывая в completed\_date.

\end{fulllineitems}

\index{from\_dict() (метод класса taskmanager.models.Task)@\spxentry{from\_dict()}\spxextra{метод класса taskmanager.models.Task}}

\begin{fulllineitems}
\phantomsection\label{\detokenize{taskmanager.models:taskmanager.models.Task.from_dict}}
\pysigstartsignatures
\pysiglinewithargsret
{\sphinxbfcode{\sphinxupquote{\DUrole{k}{classmethod}\DUrole{w}{ }}}\sphinxbfcode{\sphinxupquote{from\_dict}}}
{\sphinxparam{\DUrole{n}{data}}}
{}
\pysigstopsignatures
\sphinxAtStartPar
Метод для создания объекта из словаря.
\begin{quote}\begin{description}
\sphinxlineitem{Параметры}
\sphinxAtStartPar
\sphinxstyleliteralstrong{\sphinxupquote{data}} (\sphinxstyleliteralemphasis{\sphinxupquote{dict}}) \textendash{} Словарь, содержащий данные задачи, полученный из to\_dict().

\sphinxlineitem{Результат}
\sphinxAtStartPar
Новый объект задачи с данными из словаря.

\sphinxlineitem{Тип результата}
\sphinxAtStartPar
{\hyperref[\detokenize{taskmanager.models:taskmanager.models.Task}]{\sphinxcrossref{Task}}}

\sphinxlineitem{Исключение}
\sphinxAtStartPar
\sphinxstyleliteralstrong{\sphinxupquote{KeyError}} \textendash{} Если в словаре отсутствуют обязательные ключи.

\end{description}\end{quote}

\end{fulllineitems}

\index{to\_dict() (метод taskmanager.models.Task)@\spxentry{to\_dict()}\spxextra{метод taskmanager.models.Task}}

\begin{fulllineitems}
\phantomsection\label{\detokenize{taskmanager.models:taskmanager.models.Task.to_dict}}
\pysigstartsignatures
\pysiglinewithargsret
{\sphinxbfcode{\sphinxupquote{to\_dict}}}
{}
{}
\pysigstopsignatures
\sphinxAtStartPar
Метод для преобразования в словарь для сохранения.
\begin{quote}\begin{description}
\sphinxlineitem{Результат}
\sphinxAtStartPar
Словарь, содержащий данные задачи, где ключи \sphinxhyphen{} названия полей, значения \sphinxhyphen{} соответствующие им данные.

\sphinxlineitem{Тип результата}
\sphinxAtStartPar
dict

\end{description}\end{quote}

\end{fulllineitems}


\end{fulllineitems}


\sphinxstepscope


\subsection{taskmanager.storage module}
\label{\detokenize{taskmanager.storage:module-taskmanager.storage}}\label{\detokenize{taskmanager.storage:taskmanager-storage-module}}\label{\detokenize{taskmanager.storage::doc}}\index{module@\spxentry{module}!taskmanager.storage@\spxentry{taskmanager.storage}}\index{taskmanager.storage@\spxentry{taskmanager.storage}!module@\spxentry{module}}
\sphinxAtStartPar
В этом файле размещены функции, отвечающие за действия над задачами.
\index{Json (класс в taskmanager.storage)@\spxentry{Json}\spxextra{класс в taskmanager.storage}}

\begin{fulllineitems}
\phantomsection\label{\detokenize{taskmanager.storage:taskmanager.storage.Json}}
\pysigstartsignatures
\pysiglinewithargsret
{\sphinxbfcode{\sphinxupquote{\DUrole{k}{class}\DUrole{w}{ }}}\sphinxcode{\sphinxupquote{taskmanager.storage.}}\sphinxbfcode{\sphinxupquote{Json}}}
{\sphinxparam{\DUrole{n}{filename}}}
{}
\pysigstopsignatures
\sphinxAtStartPar
Базовые классы: \sphinxcode{\sphinxupquote{object}}

\sphinxAtStartPar
Класс служит для взаимодействия программы с json\sphinxhyphen{}файлом.
\begin{description}
\sphinxlineitem{Atributes:}
\sphinxAtStartPar
filename (str): Путь к json\sphinxhyphen{}файлу.

\sphinxlineitem{Позволяет:}
\sphinxAtStartPar
создавать,
записывать,
сохранять изменения в этот файл,

\sphinxAtStartPar
читать,
выводить список задач из файла,

\sphinxAtStartPar
создавать новый ID,
удалять задачу, перезаписывая json\sphinxhyphen{}файл.

\end{description}
\index{\_\_init\_\_() (метод taskmanager.storage.Json)@\spxentry{\_\_init\_\_()}\spxextra{метод taskmanager.storage.Json}}

\begin{fulllineitems}
\phantomsection\label{\detokenize{taskmanager.storage:taskmanager.storage.Json.__init__}}
\pysigstartsignatures
\pysiglinewithargsret
{\sphinxbfcode{\sphinxupquote{\_\_init\_\_}}}
{\sphinxparam{\DUrole{n}{filename}}}
{}
\pysigstopsignatures
\sphinxAtStartPar
Метод инициализирует атрибут класса.
\begin{quote}\begin{description}
\sphinxlineitem{Параметры}
\sphinxAtStartPar
\sphinxstyleliteralstrong{\sphinxupquote{filename}} (\sphinxstyleliteralemphasis{\sphinxupquote{str}}) \textendash{} Путь к json\sphinxhyphen{}файлу.

\end{description}\end{quote}

\end{fulllineitems}

\index{create\_a\_file\_if\_it\_does\_not\_exist() (метод taskmanager.storage.Json)@\spxentry{create\_a\_file\_if\_it\_does\_not\_exist()}\spxextra{метод taskmanager.storage.Json}}

\begin{fulllineitems}
\phantomsection\label{\detokenize{taskmanager.storage:taskmanager.storage.Json.create_a_file_if_it_does_not_exist}}
\pysigstartsignatures
\pysiglinewithargsret
{\sphinxbfcode{\sphinxupquote{create\_a\_file\_if\_it\_does\_not\_exist}}}
{}
{}
\pysigstopsignatures
\sphinxAtStartPar
Метод создаст json\sphinxhyphen{}файл с пустым списком, если он не существует.

\end{fulllineitems}

\index{creating\_a\_new\_id() (метод taskmanager.storage.Json)@\spxentry{creating\_a\_new\_id()}\spxextra{метод taskmanager.storage.Json}}

\begin{fulllineitems}
\phantomsection\label{\detokenize{taskmanager.storage:taskmanager.storage.Json.creating_a_new_id}}
\pysigstartsignatures
\pysiglinewithargsret
{\sphinxbfcode{\sphinxupquote{creating\_a\_new\_id}}}
{\sphinxparam{\DUrole{n}{tasks}}}
{}
\pysigstopsignatures
\sphinxAtStartPar
Метод создает следующее значение идентификатора.
\begin{quote}\begin{description}
\sphinxlineitem{Параметры}
\sphinxAtStartPar
\sphinxstyleliteralstrong{\sphinxupquote{tasks}} (\sphinxstyleliteralemphasis{\sphinxupquote{list}}) \textendash{} Список существующих задач.

\sphinxlineitem{Результат}
\sphinxAtStartPar
Новый уникальный идентификатор для новой задачи ИЛИ 1, если это первая задача.

\sphinxlineitem{Тип результата}
\sphinxAtStartPar
int

\end{description}\end{quote}

\end{fulllineitems}

\index{delete\_task() (метод taskmanager.storage.Json)@\spxentry{delete\_task()}\spxextra{метод taskmanager.storage.Json}}

\begin{fulllineitems}
\phantomsection\label{\detokenize{taskmanager.storage:taskmanager.storage.Json.delete_task}}
\pysigstartsignatures
\pysiglinewithargsret
{\sphinxbfcode{\sphinxupquote{delete\_task}}}
{\sphinxparam{\DUrole{n}{task\_id}}}
{}
\pysigstopsignatures
\sphinxAtStartPar
Метод удаляет задачи.
\begin{quote}\begin{description}
\sphinxlineitem{Параметры}
\sphinxAtStartPar
\sphinxstyleliteralstrong{\sphinxupquote{task\_id}} (\sphinxstyleliteralemphasis{\sphinxupquote{int}}) \textendash{} Идентификатор задачи.

\sphinxlineitem{Результат}
\sphinxAtStartPar
True, если задача успешно удалена.

\sphinxlineitem{Тип результата}
\sphinxAtStartPar
bool

\sphinxlineitem{Исключение}
\sphinxAtStartPar
\sphinxstyleliteralstrong{\sphinxupquote{ValueError}} \textendash{} Если задача не найдена по идентификатору.

\end{description}\end{quote}

\end{fulllineitems}

\index{getting\_all\_tasks() (метод taskmanager.storage.Json)@\spxentry{getting\_all\_tasks()}\spxextra{метод taskmanager.storage.Json}}

\begin{fulllineitems}
\phantomsection\label{\detokenize{taskmanager.storage:taskmanager.storage.Json.getting_all_tasks}}
\pysigstartsignatures
\pysiglinewithargsret
{\sphinxbfcode{\sphinxupquote{getting\_all\_tasks}}}
{}
{}
\pysigstopsignatures
\sphinxAtStartPar
Метод отображает все задачи.
\begin{quote}\begin{description}
\sphinxlineitem{Результат}
\sphinxAtStartPar
Список объектов Task, преобразованных из json\sphinxhyphen{}файла.

\sphinxlineitem{Тип результата}
\sphinxAtStartPar
list

\end{description}\end{quote}

\end{fulllineitems}

\index{load\_tasks() (метод taskmanager.storage.Json)@\spxentry{load\_tasks()}\spxextra{метод taskmanager.storage.Json}}

\begin{fulllineitems}
\phantomsection\label{\detokenize{taskmanager.storage:taskmanager.storage.Json.load_tasks}}
\pysigstartsignatures
\pysiglinewithargsret
{\sphinxbfcode{\sphinxupquote{load\_tasks}}}
{}
{}
\pysigstopsignatures
\sphinxAtStartPar
Метод считывает информацию о существующих задачах из json\sphinxhyphen{}файла.
\begin{quote}\begin{description}
\sphinxlineitem{Результат}
\sphinxAtStartPar
Список словарей с информацией о задачах из json\sphinxhyphen{}файла ИЛИ пустой список в случае ошибки.

\sphinxlineitem{Тип результата}
\sphinxAtStartPar
list

\end{description}\end{quote}

\end{fulllineitems}

\index{save\_tasks() (метод taskmanager.storage.Json)@\spxentry{save\_tasks()}\spxextra{метод taskmanager.storage.Json}}

\begin{fulllineitems}
\phantomsection\label{\detokenize{taskmanager.storage:taskmanager.storage.Json.save_tasks}}
\pysigstartsignatures
\pysiglinewithargsret
{\sphinxbfcode{\sphinxupquote{save\_tasks}}}
{\sphinxparam{\DUrole{n}{task}}}
{}
\pysigstopsignatures
\sphinxAtStartPar
Метод записывает или обновляет информацию о текущей задачи в json\sphinxhyphen{}файл.
Если задача существует, то производит обновление данных об этой задаче.
\begin{quote}\begin{description}
\sphinxlineitem{Параметры}
\sphinxAtStartPar
\sphinxstyleliteralstrong{\sphinxupquote{task}} ({\hyperref[\detokenize{taskmanager.models:taskmanager.models.Task}]{\sphinxcrossref{\sphinxstyleliteralemphasis{\sphinxupquote{Task}}}}}) \textendash{} Объект задачи для сохранения.

\sphinxlineitem{Результат}
\sphinxAtStartPar
Сохраненная задача с обновлёнными данными. Новые задачи вернутся с присвоенным ID.

\sphinxlineitem{Тип результата}
\sphinxAtStartPar
{\hyperref[\detokenize{taskmanager.models:taskmanager.models.Task}]{\sphinxcrossref{Task}}}

\end{description}\end{quote}

\end{fulllineitems}

\index{write\_tasks() (метод taskmanager.storage.Json)@\spxentry{write\_tasks()}\spxextra{метод taskmanager.storage.Json}}

\begin{fulllineitems}
\phantomsection\label{\detokenize{taskmanager.storage:taskmanager.storage.Json.write_tasks}}
\pysigstartsignatures
\pysiglinewithargsret
{\sphinxbfcode{\sphinxupquote{write\_tasks}}}
{\sphinxparam{\DUrole{n}{tasks}}}
{}
\pysigstopsignatures
\sphinxAtStartPar
Запись изменений в json\sphinxhyphen{}файл.
\begin{quote}\begin{description}
\sphinxlineitem{Параметры}
\sphinxAtStartPar
\sphinxstyleliteralstrong{\sphinxupquote{tasks}} (\sphinxstyleliteralemphasis{\sphinxupquote{list}}) \textendash{} Список задач для записи их в json\sphinxhyphen{}файл.

\end{description}\end{quote}

\end{fulllineitems}


\end{fulllineitems}


\sphinxstepscope


\subsection{taskmanager.commands module}
\label{\detokenize{taskmanager.commands:module-taskmanager.commands}}\label{\detokenize{taskmanager.commands:taskmanager-commands-module}}\label{\detokenize{taskmanager.commands::doc}}\index{module@\spxentry{module}!taskmanager.commands@\spxentry{taskmanager.commands}}\index{taskmanager.commands@\spxentry{taskmanager.commands}!module@\spxentry{module}}
\sphinxAtStartPar
В этом файле содержатся обработчики команд для командной строки main.py.
\index{Command (класс в taskmanager.commands)@\spxentry{Command}\spxextra{класс в taskmanager.commands}}

\begin{fulllineitems}
\phantomsection\label{\detokenize{taskmanager.commands:taskmanager.commands.Command}}
\pysigstartsignatures
\pysiglinewithargsret
{\sphinxbfcode{\sphinxupquote{\DUrole{k}{class}\DUrole{w}{ }}}\sphinxcode{\sphinxupquote{taskmanager.commands.}}\sphinxbfcode{\sphinxupquote{Command}}}
{\sphinxparam{\DUrole{n}{storage}\DUrole{p}{:}\DUrole{w}{ }\DUrole{n}{{\hyperref[\detokenize{taskmanager.storage:taskmanager.storage.Json}]{\sphinxcrossref{Json}}}}\DUrole{w}{ }\DUrole{o}{=}\DUrole{w}{ }\DUrole{default_value}{None}}}
{}
\pysigstopsignatures
\sphinxAtStartPar
Базовые классы: \sphinxcode{\sphinxupquote{object}}

\sphinxAtStartPar
Класс служит для управления задачами через командную строку.
\index{storage (атрибут taskmanager.commands.Command)@\spxentry{storage}\spxextra{атрибут taskmanager.commands.Command}}

\begin{fulllineitems}
\phantomsection\label{\detokenize{taskmanager.commands:taskmanager.commands.Command.storage}}
\pysigstartsignatures
\pysigline
{\sphinxbfcode{\sphinxupquote{storage}}}
\pysigstopsignatures
\sphinxAtStartPar
Объект для работы с хранилищем задач.

\end{fulllineitems}

\index{\_\_init\_\_() (метод taskmanager.commands.Command)@\spxentry{\_\_init\_\_()}\spxextra{метод taskmanager.commands.Command}}

\begin{fulllineitems}
\phantomsection\label{\detokenize{taskmanager.commands:taskmanager.commands.Command.__init__}}
\pysigstartsignatures
\pysiglinewithargsret
{\sphinxbfcode{\sphinxupquote{\_\_init\_\_}}}
{\sphinxparam{\DUrole{n}{storage}\DUrole{p}{:}\DUrole{w}{ }\DUrole{n}{{\hyperref[\detokenize{taskmanager.storage:taskmanager.storage.Json}]{\sphinxcrossref{Json}}}}\DUrole{w}{ }\DUrole{o}{=}\DUrole{w}{ }\DUrole{default_value}{None}}}
{}
\pysigstopsignatures
\sphinxAtStartPar
Метод инициализирует атрибут класса.
\begin{quote}\begin{description}
\sphinxlineitem{Параметры}
\sphinxAtStartPar
\sphinxstyleliteralstrong{\sphinxupquote{storage}} \textendash{} Объект хранилища задач. Если не указан, используется Json(«tasks.json»).

\end{description}\end{quote}

\end{fulllineitems}

\index{add\_task() (метод taskmanager.commands.Command)@\spxentry{add\_task()}\spxextra{метод taskmanager.commands.Command}}

\begin{fulllineitems}
\phantomsection\label{\detokenize{taskmanager.commands:taskmanager.commands.Command.add_task}}
\pysigstartsignatures
\pysiglinewithargsret
{\sphinxbfcode{\sphinxupquote{add\_task}}}
{\sphinxparam{\DUrole{n}{title}}\sphinxparamcomma \sphinxparam{\DUrole{n}{description}}\sphinxparamcomma \sphinxparam{\DUrole{n}{priority}}\sphinxparamcomma \sphinxparam{\DUrole{n}{due\_date}}}
{}
\pysigstopsignatures
\sphinxAtStartPar
Метод добавляет новую задачу.
\begin{quote}\begin{description}
\sphinxlineitem{Параметры}\begin{itemize}
\item {} 
\sphinxAtStartPar
\sphinxstyleliteralstrong{\sphinxupquote{title}} \textendash{} Название задачи.

\item {} 
\sphinxAtStartPar
\sphinxstyleliteralstrong{\sphinxupquote{description}} \textendash{} Описание задачи.

\item {} 
\sphinxAtStartPar
\sphinxstyleliteralstrong{\sphinxupquote{priority}} \textendash{} Приоритет задачи (low, medium, high).

\item {} 
\sphinxAtStartPar
\sphinxstyleliteralstrong{\sphinxupquote{due\_date}} \textendash{} Дата выполнения в формате YYYY\sphinxhyphen{}MM\sphinxhyphen{}DD.

\end{itemize}

\sphinxlineitem{Результат}
\sphinxAtStartPar
Результат операции сохранения.

\sphinxlineitem{Тип результата}
\sphinxAtStartPar
bool

\end{description}\end{quote}

\end{fulllineitems}

\index{complete\_task() (метод taskmanager.commands.Command)@\spxentry{complete\_task()}\spxextra{метод taskmanager.commands.Command}}

\begin{fulllineitems}
\phantomsection\label{\detokenize{taskmanager.commands:taskmanager.commands.Command.complete_task}}
\pysigstartsignatures
\pysiglinewithargsret
{\sphinxbfcode{\sphinxupquote{complete\_task}}}
{\sphinxparam{\DUrole{n}{task\_id}}}
{}
\pysigstopsignatures
\sphinxAtStartPar
Метод отмечает задачу, как выполненную.
\begin{quote}\begin{description}
\sphinxlineitem{Параметры}
\sphinxAtStartPar
\sphinxstyleliteralstrong{\sphinxupquote{task\_id}} \textendash{} Идентификатор задачи для изменения статуса.

\sphinxlineitem{Результат}
\sphinxAtStartPar
True если операция выполнена успешно.

\sphinxlineitem{Тип результата}
\sphinxAtStartPar
bool

\sphinxlineitem{Исключение}
\sphinxAtStartPar
\sphinxstyleliteralstrong{\sphinxupquote{ValueError}} \textendash{} Если задача с указанным ID не найдена.

\end{description}\end{quote}

\end{fulllineitems}

\index{delete\_task() (метод taskmanager.commands.Command)@\spxentry{delete\_task()}\spxextra{метод taskmanager.commands.Command}}

\begin{fulllineitems}
\phantomsection\label{\detokenize{taskmanager.commands:taskmanager.commands.Command.delete_task}}
\pysigstartsignatures
\pysiglinewithargsret
{\sphinxbfcode{\sphinxupquote{delete\_task}}}
{\sphinxparam{\DUrole{n}{task\_id}}}
{}
\pysigstopsignatures
\sphinxAtStartPar
Метод удаляет задачу.
\begin{quote}\begin{description}
\sphinxlineitem{Параметры}
\sphinxAtStartPar
\sphinxstyleliteralstrong{\sphinxupquote{task\_id}} \textendash{} Идентификатор задачи для удаления.

\sphinxlineitem{Результат}
\sphinxAtStartPar
Результат операции удаления.

\sphinxlineitem{Тип результата}
\sphinxAtStartPar
bool

\end{description}\end{quote}

\end{fulllineitems}

\index{filter\_task() (метод taskmanager.commands.Command)@\spxentry{filter\_task()}\spxextra{метод taskmanager.commands.Command}}

\begin{fulllineitems}
\phantomsection\label{\detokenize{taskmanager.commands:taskmanager.commands.Command.filter_task}}
\pysigstartsignatures
\pysiglinewithargsret
{\sphinxbfcode{\sphinxupquote{filter\_task}}}
{\sphinxparam{\DUrole{n}{status}\DUrole{o}{=}\DUrole{default_value}{None}}\sphinxparamcomma \sphinxparam{\DUrole{n}{priority}\DUrole{o}{=}\DUrole{default_value}{None}}\sphinxparamcomma \sphinxparam{\DUrole{n}{due\_date}\DUrole{o}{=}\DUrole{default_value}{None}}\sphinxparamcomma \sphinxparam{\DUrole{n}{filter\_flag}\DUrole{o}{=}\DUrole{default_value}{False}}}
{}
\pysigstopsignatures
\sphinxAtStartPar
Метод возвращает отфильтрованный список.
\begin{quote}\begin{description}
\sphinxlineitem{Параметры}\begin{itemize}
\item {} 
\sphinxAtStartPar
\sphinxstyleliteralstrong{\sphinxupquote{status}} \textendash{} Статус задачи для фильтрации («Ожидание», «Выполнено»).

\item {} 
\sphinxAtStartPar
\sphinxstyleliteralstrong{\sphinxupquote{priority}} \textendash{} Приоритет задачи для фильтрации («low», «medium», «high»).

\item {} 
\sphinxAtStartPar
\sphinxstyleliteralstrong{\sphinxupquote{due\_date}} \textendash{} Дата выполнения для фильтрации.

\item {} 
\sphinxAtStartPar
\sphinxstyleliteralstrong{\sphinxupquote{filter\_flag}} \textendash{} Если True, скрывает выполненные задачи.

\end{itemize}

\sphinxlineitem{Результат}
\sphinxAtStartPar
Отфильтрованный список объектов Task.

\sphinxlineitem{Тип результата}
\sphinxAtStartPar
list

\end{description}\end{quote}

\end{fulllineitems}

\index{get\_task() (метод taskmanager.commands.Command)@\spxentry{get\_task()}\spxextra{метод taskmanager.commands.Command}}

\begin{fulllineitems}
\phantomsection\label{\detokenize{taskmanager.commands:taskmanager.commands.Command.get_task}}
\pysigstartsignatures
\pysiglinewithargsret
{\sphinxbfcode{\sphinxupquote{get\_task}}}
{\sphinxparam{\DUrole{n}{task\_id}}}
{}
\pysigstopsignatures
\sphinxAtStartPar
Метод возвращает информацию о задаче по её идентификатору.
\begin{quote}\begin{description}
\sphinxlineitem{Параметры}
\sphinxAtStartPar
\sphinxstyleliteralstrong{\sphinxupquote{task\_id}} \textendash{} Идентификатор задачи.

\sphinxlineitem{Результат}
\sphinxAtStartPar
Если объект задачи найден.
None: Если Объект задачи не найден.

\sphinxlineitem{Тип результата}
\sphinxAtStartPar
{\hyperref[\detokenize{taskmanager.models:taskmanager.models.Task}]{\sphinxcrossref{Task}}}

\end{description}\end{quote}

\end{fulllineitems}


\end{fulllineitems}

\index{setup\_parser() (в модуле taskmanager.commands)@\spxentry{setup\_parser()}\spxextra{в модуле taskmanager.commands}}

\begin{fulllineitems}
\phantomsection\label{\detokenize{taskmanager.commands:taskmanager.commands.setup_parser}}
\pysigstartsignatures
\pysiglinewithargsret
{\sphinxcode{\sphinxupquote{taskmanager.commands.}}\sphinxbfcode{\sphinxupquote{setup\_parser}}}
{}
{}
\pysigstopsignatures
\sphinxAtStartPar
Метод создает команды для командной строки.
\begin{quote}\begin{description}
\sphinxlineitem{Результат}
\sphinxAtStartPar
Настроенный парсер аргументов.

\sphinxlineitem{Тип результата}
\sphinxAtStartPar
argparse.ArgumentParser

\end{description}\end{quote}

\end{fulllineitems}



\renewcommand{\indexname}{Содержание модулей Python}
\begin{sphinxtheindex}
\let\bigletter\sphinxstyleindexlettergroup
\bigletter{t}
\item\relax\sphinxstyleindexentry{taskmanager}\sphinxstyleindexpageref{taskmanager:\detokenize{module-taskmanager}}
\item\relax\sphinxstyleindexentry{taskmanager.commands}\sphinxstyleindexpageref{taskmanager.commands:\detokenize{module-taskmanager.commands}}
\item\relax\sphinxstyleindexentry{taskmanager.models}\sphinxstyleindexpageref{taskmanager.models:\detokenize{module-taskmanager.models}}
\item\relax\sphinxstyleindexentry{taskmanager.storage}\sphinxstyleindexpageref{taskmanager.storage:\detokenize{module-taskmanager.storage}}
\end{sphinxtheindex}

\renewcommand{\indexname}{Алфавитный указатель}
\printindex
\end{document}